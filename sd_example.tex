% vim: spell
\documentclass{ocesd}

\begin{document}

\name{Student name}

\banner{bannerID}

\startdate{start-date} % insert the date of arrival in program

\mip{MIP} % Insert the number of months in the program

\department{Oceanography} % Insert name of department

\program{PhD} % Insert MSc or PhD

\phdtransfer{} % fill in the transfer date for a switch from MSc to PhD, if appropriate

\supervisor{supervisor} % name the supervisor

\cosupervisor{} % insert name of cosupervisor, or empty

\committee{...} % list the committee members

\meetingdates{...} % list all meetings, in YYYY-MM-DD [MIP] 

\funding{List funding sources, e.g. scholarships, titles of grants to
supervisor, etc}

\scholarships{List any scholarships held or previously held, with dates of
tenure, funding agency, etc}

\coursestaken{...} % list courses taken with course numbers and names, along with dates and grades

\coursesplanned{...} % list courses planned, with numbers and names

\thesisproposal{state time completed or planned}

\qualifyingexamination{PhD students who have completed the qualifying
examination should state the date of the examination, the outcome, the
examination committee (noting the chair and the external examiner), and the
assigned papers.  PhD students who have not yet completed the examination
should state the planned date, if known.  MSc students should leave this
blank.}

\fieldwork{State the nature and location of the fieldwork, the date completed
(or planned).}

\thesis{Provide information on the thesis as is appropriate to the stage in the
program.  Students should be able to state a provisional title near the start
of the program.  By the halfway point, students should e able to state
provisional titles of chapters, and often the degree of completion of those
chapters.  Students in the last quarter of the program should be able to sketch
the timetable for completion of the work and the writing.  All students should
state an expected date for the thesis defence.}

\presentations{List any presentations given, with dates, venues, titles, and
any other information that seems useful (e.g. formal vs informal, invited vs
uninvited, poster vs oral).}

\publications{List any publications in preparation, submitted, in press, or
published.  Provide standard information, including dates, titles, coauthors,
journals, whether material is peer-reviewed, etc.}

\awards{List any awards that relate to the academic work, with dates, titles,
and agencies.  Optionally, list also any non-academic awards (e.g. for
volunteer activities).}

\teaching{List contributions to teaching, including TA work, guest lectures in
classes, etc., providing dates and titles as appropriate.}

\contributions{List (a) contributions to the department, (b) contributions to
the university, and (c) contributions to the community.  Provide dates and the
level of detail you deem to be useful for your other purposes.}

\otheraccomplishments{List any other accomplishments that seem relevant (e.g.
that might be useful in another CV you might develop for applying for a job or
scholarship in the future).}

\discussion{Optionally: use this space to list any discussion points you would
like to raise in a meeting with your supervisor or the graduate coordinator.}

\maketitle

\end{document}
